\section{Windfarm problem implementation}
    \subsection{Summary}
    This implementation of the windfarm maintenance problem is done using GLPK/GMPL
    (GNU Linear Programming Kit and Modelling language GNU MathProg). The program is broken down to
    3 input and 2 output files:
    \begin{itemize}
        \item Input files
            \begin{itemize}
                \item \textit{windfarm.mod}: model file containing the core of the modelling problem.
                Declares sets, parameters, conditions and the objective of the modelling exercise.
                Also, the file includes \textit{printf} statements that generates more readable output
                than the standard built-in one in \textit{GLPK}. This customized output is available in 
                the \textit{report.txt} file.
                \item \textit{wfmaintenance.dat}: data file with initialization of sets and parameters 
                related to maintenance jobs. Additionally, the parameter matrix \textit{main\_req\_st} 
                that links maintenance tasks and staff related parameters is also initialized here.
                Following parameters are initialized here:
                \begin{itemize}
                    \item set MaintenanceTypes
                    \item set MaintenanceSeverity
                    \item param main\_req
                    \item param main\_req\_xp
                    \item param main\_req\_st
                    \item param main\_burnout
                    \item param burnout\_coef
                    \item param main\_material\_cost
                \end{itemize}
                \item \textit{wfstaff.dat}: data file initializing staff related sets and parameters.
                \begin{itemize}
                    \item set StaffTypes
                    \item set StaffLevels
                    \item param staff\_level\_xp
                    \item param staff\_cost
                \end{itemize}
            \end{itemize}
        \item Output files
            \begin{itemize}
                \item \textit{output.txt}: standard output file generated by the \textit{glpsol} 
                command with the \textit{-o} parameter. The file shows the values of all variables 
                and the values used for resolving the conditions specified in the .mod file.
                \item \textit{report.txt}: customized output generated by the \textit{printf} 
                statements at the end of the \textit{windfarm.mod} file. Contentwise same as 
                \textit{output.txt}.
            \end{itemize}
    \end{itemize}

    \subsection{Model setup}

        \subsubsection{Conditions and variables}
        The condition section of the modelling file is broken down to 4 main sections, XP conditions,
        Staff conditions, Minimum maintenance and Burnout indicators respectively. The XP condition 
        takes care of that the hired staff have at least as much XP points that is necessary to be able
        to carry out the required maintenance tasks. The following variables are calculated within this
        section: 
        \newpage
        
        \subparagraph{XP conditions} \label{xp condition} \text{} \newline
        \hspace*{.5cm}Firstly, redundant variables:
        %total_main_req_xp[mt] = (if main_req[mt,"severe"] <> 0 then main_req_xp[mt,"severe"] else main_req_xp[mt,"normal"]);
        \begin{equation}
            %\hfilneg 
            \underset{mt\times 1}{\mathrm{\text{total\_main\_req\_xp}}} = 
            \begin{cases}
                \underset{mt\times ms}{\text{main\_req\_xp[mt,"severe"]}}, \underset{mt\times ms}{\text{ if } \text{main\_req\_xp[mt,"severe"]}} > 0\\[3ex]
                \underset{mt\times ms}{\text{main\_req\_xp[mt,"normal"]}}, \text{ otherwise}\\
            \end{cases}
            %\hspace{10000pt minus 1fil}
        \end{equation}

        %total_staff_xp[st] = sum{sl in StaffLevels} staff_to_hire[st,sl] * staff_level_xp[sl];
        \begin{equation}
            %\hfilneg
            \underset{st\times 1}{\text{total\_staff\_xp}} =  
                \underset{st\times sl}{\text{staff\_to\_hire}}\times \underset{sl\times 1}{\text{staff\_level\_xp}}
            %\hspace{10000pt minus 1fil}
        \end{equation}

        %total_staff_xp_task[mt] = sum{st in StaffTypes} if main_req_st[mt,st] <> 0 then total_staff_xp[st] else 0 ;
        \begin{equation}
            %\hfill
            \underset{mt\times 1}{\text{total\_staff\_xp\_task}} =
            \begin{cases}
                \underset{st\times 1}{\text{total\_staff\_xp[st]}}, \underset{mt\times st}{\text{ if } \text{main\_req\_st[mt,st]}} > 0\\[3ex]
                0, \text{ otherwise}
            \end{cases}
            %\hspace{10000pt minus 1fil}
        \end{equation} \newline

        \text{Finally, the required condition is}
        %total_staff_xp_task[mt] >= total_main_req_xp[mt];
        \begin{equation}
            \underset{mt\times 1}{\text{total\_staff\_xp\_task}} >=  
                \underset{mt\times 1}{\text{total\_main\_req\_xp}}
        \end{equation}

        \subparagraph{Staff conditions} \label{staff condition} \text{} \newline
        \hspace*{.5cm}Again, introduce (this time only one) redundant variable:
        %total_staff[st] = sum{sl in StaffLevels} staff_to_hire[st,sl];
        \begin{equation}
            \underset{st\times 1}{\text{total\_staff}} = 
                \sum_{sl}\underset{st\times sl}{\text{staff\_to\_hire[st,sl]}}
        \end{equation}

        \text{The staff condition is}
        %total_staff[st] >= main_req_st[mt, st];
        \begin{equation}
            \underset{st\times 1}{\text{total\_staff}} >= 
                \sum_{mt}\underset{mt\times st}{\text{main\_req\_st}^T\text{[mt,st]}}
        \end{equation}

        \subparagraph{Minimum maintenance} \label{minimum maintenance} \text{} \newline
        \hspace*{.5cm}No redundant variables this time:
        %quantity[mt] >= sum{ms in MaintenanceSeverity} main_req[mt,ms];
        \begin{equation}
            \underset{mt\times 1}{\text{quantity}} >= 
                \sum_{ms}\underset{mt\times ms}{\text{main\_req[mt,ms]}}
        \end{equation}

        \subparagraph{Burnout condition} \label{burnout condition} \text{} \newline
        \hspace*{.5cm}Redundant variables:
        %weighted_maintenance_tasks[mt] = sum{ms in MaintenanceSeverity} main_req[mt,ms] * main_burnout[ms];
        \begin{equation}
            %\hfilneg
            \underset{mt\times 1}{\text{weighted\_maintenance\_tasks}} =  
                \underset{mt\times ms}{\text{main\_req[mt,ms]}}\times \underset{ms\times 1}{\text{main\_burnout[ms]}}
            %\hspace{10000pt minus 1fil}
        \end{equation}

        %total_req_wgt_staff[st] = sum{mt in MaintenanceTypes} weighted_maintenance_tasks[mt] * main_req_st[mt,st];
        \begin{equation}
            %\hfilneg
            \underset{1\times st}{\text{total\_req\_wgt\_staff}} =  
                \underset{mt\times 1}{{\text{weighted\_maintenance\_tasks}^T\text{[mt]}}}\times \underset{mt\times st}{\text{main\_req\_st[mt,st]}}
            %\hspace{10000pt minus 1fil}
        \end{equation}

        The required burnout condition is:
        %total_req_wgt_staff[st] <= total_staff[st] * burnout_coef;
        \begin{equation}
            \underset{st\times 1}{\text{total\_req\_wgt\_staff[st]}} <= 
                \underset{st\times 1}{\text{total\_staff[st]}} * \text{burnout\_coef}
        \end{equation}

        where \newline
            \hspace*{.75cm} \textit{mt} is in MaintenanceTypes \newline
            \hspace*{.75cm} \textit{ms} is in MaintenanceSeverity \newline
            \hspace*{.75cm} \textit{st} is in StaffTypes \newline
            \hspace*{.75cm} \textit{st} is in StaffLevels \newline
        \hspace*{.55cm}sets.

        \subsubsection{Objective}
        The objective function of the modelling exercise is minimizing total costs obtained as the 
        sum of material and staff related costs while fulfilling required maintenance tasks given 
        the conditions. Total staff cost is obtained by multiplying matrices 
        \textit{staff\_to\_hire} and \textit{staff\_cost} and then taking the trace of the resulting
        matrix:
        \[
            \text{TotalStaffCost} = 
            \text{tr}\left( \underset{(st \times sl)}{\text{staff\_to\_hire}} \times
            \underset{(st \times sl)}{\text{staff\_cost}^T}\right)
        \]

        Total material cost can be obtained by simply taking the sumproduct of vectors
        \textit{main\_material\_cost} and \textit{quantity}:
        \[
            \text{TotalMaterialCost} = 
            \sum_{mt}\text{main\_maintenance\_cost[mt]} * \text{quantity[mt]}
        \]

        Objective to be minimized:
        \begin{equation}
            \text{minimize TotalCosts} = 
                \text{TotalStaffCost} + \text{TotalMaterialCost}
        \end{equation}

    \subsection{Sample output}
        \subsubsection{Parameter main\_req}
        The parameter contains the required number of maintenance jobs (rows) based on severity 
        (columns). The required number of maintenance jobs per types is given by the row sums 
        (summed over severity level). The row sums are used in the condition \nameref{minimum maintenance}.
        If the "severe" column for a maintenance job is greater than 0, it
        means the 
        \begin{table}[ht]
            \begin{center}
                \begin{tabular}{ |c|c|c| } 
                \hline
                Maintenance Type & normal & severe \\ 
                \hline
                blades    & 12 & 5  \\ 
                gearbox   & 13 & 3  \\ 
                generator & 6  & 10 \\ 
                sensors   & 8  & 1  \\ 
                wiring    & 9  & 0  \\ 
                \hline
                \end{tabular}
            \end{center}
        \caption{caption} % displayed in doc
        \label{param_main_req} % for reference
        \end{table}

    \subsection{Effect of parameters}