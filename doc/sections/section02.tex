\section{Windfarm problem implementation}
    \subsection{Summary}
    This implementation of the windfarm maintenance problem is done using GLPK/GMPL
    (GNU Linear Programming Kit and Modelling language GNU MathProg). The program is broken down to
    3 input and 2 output files:
    \begin{itemize}
        \item Input files
            \begin{itemize}
                \item \textit{windfarm.mod}: model file containing the core of the modelling problem.
                Declares sets, parameters, conditions and the objective of the modelling exercise.
                Also, the file includes \textit{printf} statements that generates more readable output
                than the standard built-in one in \textit{GLPK}. This customized output is available in 
                the \textit{report.txt} file.
                \item \textit{wfmaintenance.dat}: data file with initialization of sets and parameters 
                related to maintenance jobs. Additionally, the parameter matrix \textit{main\_req\_st} 
                that links maintenance tasks and staff related parameters is also initialized here.
                Following parameters are initialized here:
                \begin{itemize}
                    \item set MaintenanceTypes
                    \item set MaintenanceSeverity
                    \item param main\_req
                    \item param main\_req\_xp
                    \item param main\_req\_st
                    \item param main\_burnout
                    \item param burnout\_coef
                    \item param main\_material\_cost
                \end{itemize}
                \item \textit{wfstaff.dat}: data file initializing staff related sets and parameters.
                \begin{itemize}
                    \item set StaffTypes
                    \item set StaffLevels
                    \item param staff\_level\_xp
                    \item param staff\_cost
                \end{itemize}
            \end{itemize}
        \item Output files
            \begin{itemize}
                \item \textit{output.txt}: standard output file generated by the \textit{glpsol} 
                command with the \textit{-o} parameter. The file shows the values of all variables 
                and the values used for resolving the conditions specified in the .mod file.
                \item \textit{report.txt}: customized output generated by the \textit{printf} 
                statements at the end of the \textit{windfarm.mod} file. Contentwise same as 
                \textit{output.txt}.
            \end{itemize}
    \end{itemize}

    \subsection{Model setup}
        \subsubsection{Sample Matrix}
            This is just a sample matrix\\
            \begin{gather}
                \begin{bmatrix} \Phi_{11} & \Phi_{12} \\ \Phi_{21} & \Phi_{22} \end{bmatrix}
                =
                \frac{1}{\det(X)}
                \begin{bmatrix}
                X_{22} Y_{11} - X_{12} Y_{21} &
                X_{22} Y_{12} - X_{12} Y_{22} \\
                X_{11} Y_{21} - X_{21} Y_{11} &
                X_{11} Y_{22} - X_{21} Y_{12} 
                \end{bmatrix}
            \end{gather}


        \subsubsection{Conditions and variables}
        The condition section of the modelling file is broken down to 4 main sections, XP conditions,
        Staff conditions, Minimum maintenance and Burnout indicators respectively. The XP condition 
        takes care of that the hired staff have at least as much XP points that is necessary to be able
        to carry out the required maintenance tasks. The following variables are calculated within this
        section:\\
        
        \textbf{XP conditions} \newline
        \begin{equation}
            \underset{mt\times 1}{\mathrm{\text{total\_main\_req\_xp}}} = 
            \begin{cases}
                \underset{mt\times ms}{\text{main\_req\_xp[mt,"severe"]}}, \underset{mt\times ms}{\text{ if } \text{main\_req\_xp[mt,"severe"]}} > 0\\
                \underset{mt\times ms}{\text{main\_req\_xp[mt,"normal"]}}, \text{ otherwise}\\
            \end{cases}
        \end{equation}

        \begin{equation}
            \underset{st\times 1}{\text{total\_staff\_xp}} =  
                \underset{st\times sl}{\text{staff\_to\_hire}}\times \underset{sl\times 1}{\text{staff\_level\_xp}}
        \end{equation}

        \textbf{Staff conditions} \newline

        \textbf{Minimum maintenance} \newline

        \textbf{Burnout condition} \newline


        where \newline
            \indent \textit{mt} is in MaintenanceTypes \newline
            \indent \textit{ms} is in MaintenanceSeverity \newline
            \indent \textit{st} is in StaffTypes \newline
            \indent \textit{st} is in StaffLevels \newline
        sets.


    \subsection{Sample output}


    \subsection{Effect of parameters}